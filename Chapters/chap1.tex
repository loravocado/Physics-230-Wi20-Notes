
\chapter{Electric Charge and Electric Field}

\section{Electric Charge}
\textbf{Electromagnetism} (EM) affects only charged particles, mainly electrons and protons. All particles have charges that are integer multiples of the elementary charge $e$ such that the charge is given by
\begin{equation}
q = ne,
\end{equation}
where $q$ represents charge (C), $n$ represents an integer, and $e = \SI{1.6022e-19}{C}$ represents the elementary charge.

Electric charge is conserved. This means that the total charge of any isolated system with no charge moving in or out stays the same -- charge is never created or destroyed. 

\textbf{Coulomb's Law} states that charges of the same sign repel and charges of the opposite sign attract. Furthermore, the force $F$ produced by charges can be calculated via
\begin{equation}
F = k\frac{q_1q_2}{r^2},
\end{equation}
where $k = \SI{9.0e9}{\frac{N m^2}{C^2}}$, $q_1$ represents the first charge, $q_2$ represents the second charge, and $r$ represents the separation distance. Note that $k$ can also be represtented as 
\begin{equation}
    k = \dfrac{1}{4\pi\epsilon_0}.
\end{equation}

The \textbf{principle of superposition of forces} is the vector sum of the individual forces:
\begin{equation}
F = k\frac{q_1q_2}{(r_{12})^2}r_{12} + k\frac{q_1q_2}{(r_{13})^2}r_{13} = q_1 \vec{E}.
\end{equation}

\section{Electric Field and Electric Forces}
The \textbf{electric field of a point charge} at $\vec{r}$ is given by
\begin{equation}
\vec E (\vec r)= \frac{kq}{r^3}\vec{r} = \frac{\vec{F}}{q},
\end{equation}
where $q$ denotes the charge of the point source (C), $r$ denotes the radial distance of the point charge from the origin, and $\vec{r}$ represents the displacement vector of the point from the origin. We can also calculate the magnitude of the eletric field via
\begin{equation}
E = k\dfrac{\abs{q}}{r^2}.
\end{equation}

The \textbf{electric field of a group of charges} is the superposition of all the electric forces from all the charges. This can be approximated with volume charge density ($\rho$) via
\begin{equation}
\vec{E}(\vec{r_o}) = \frac{1}{4\pi \epsilon_0} \int\frac{\rho(\vec{r}) (\vec{r_0} - \vec{r})}{\abs{\vec{r_0} - \vec{r}}^3} \di x \di y \di z,
\end{equation}
where $\epsilon_0 = \SI{8.854e-12}{\frac{C^2}{N \cdot m^2}}$, $\rho(\vec{r})$ is the surface charge density, and $\vec{r}$ is the radius.

For \textbf{a conductor without current flow}, the charge all resides on the surface. Using the surface charge density $\sigma (\vec{r})$, we can approximate the electric field via
\begin{equation}
\vec{E}(\vec{r_o}) = \frac{1}{4\pi \epsilon_0} \int\frac{\sigma(\vec{r}) (\vec{r_0} - \vec{r})}{\abs{\vec{r_0} - \vec{r}}^3} \di A.
\end{equation}

If one has a \textbf{thin wire} where all the charge resides, with linear charge density $\lambda(\vec{r}) = \frac{\di q}{\di l}$, where $\di l$ is the element of length along the wire, then
\begin{equation}
\vec{E}(\vec{r_o}) = \frac{1}{4\pi \epsilon_0} \int\frac{\lambda(\vec{r}) (\vec{r_0} - \vec{r})}{\abs{\vec{r_0} - \vec{r}}^3} \di l.
\end{equation}

\textbf{Example.} For a ring of total charge $Q$ on a circle of radius $a$, the circumference is $2\pi a$ and therefore the linear charge density is $\lambda = \frac{Q}{2\pi a}$. The electric field is therefore
\begin{align}
    \vec{E}(\vec{r_0}) &= \frac{1}{4\pi\epsilon_0} \int \frac{\vec{r_0} - \vec{r}}{\abs{\vec{r_0}-\vec{r}}^3} \lambda(\vec{r}) \di l  
    = E_x \hat{i} \\
    &= \frac{1}{4\pi\epsilon} \frac{Qx}{(x^2+a^2)^{3/2}}\hat{i} \\
    &= \frac{Q \hat{i}}{4\pi\epsilon_0x^2} \quad \text{(For $x >> a$)}.
\end{align}

\section{Electric Field Lines}
Electric field lines have tangent vectors parallel to the electric field and begin only on positive charges and end only on negative charges, though they can also go to infinity in either direction.

\section{Electric Dipoles}
An \textbf{electric dipole} is a pair of point charges of equal magnitude and opposite sign (a positive charge $q$ and a negative charge $-q$ separated by a distance $d$.

An \textbf{electric dipole moment} is the product of the positive charge $q$ and the displacement $d$ it is separated from the negative charge $-q$, given by
\begin{equation}
\vec{p} = q\vec{d}.
\end{equation}

Using the volume charge density formula $\rho(\vec{r}) = \rho(x\vec{i} + y\vec{j} + z\vec{k}) = \rho(x, y, z)$, we can approximate the electric dipole of a huge number of elementary charges (total charge is neutral) via
\begin{equation}
\vec{P} = \int\rho(\vec{r})\vec{r}\di x \di y \di z.
\end{equation}

The total force on an electric dipole is just the net force from an external electric field (the electric field from all the other charges that are not part of the electric dipole). This is true for the total torque.

In a uniform electric field $\vec{E}$, the net force on a electric dipole is 0. However, if the electric dipole monent vector $\vec{p}$ is not parallel to $\vec{E}$, then a torque is exerted on the dipole that changes its angular momentum $\vec{L}$ by $\frac{\di \vec{L}}{\di t} = \vec{\tau}$. This vector torque is calculated via
\begin{equation}
\vec{\tau} = \vec{P} \times \vec{E},
\end{equation}
where $\vec{P}$ is the electric dipole moment, $\vec{E}$ is the electric field, and the direction of $\tau$ is perpendicular to both $\vec{P}$ and $\vec{E}$. The magnitude of torque can also be found via
\begin{equation}
    \tau = pE\sin\phi,
\end{equation}
where $p$ is the magnitude of the electric dipole moment $\vec{p}$, $E$ is the magnitude of the electric field $\vec{E}$, and $\phi$ is the angle between $\vec p$ and $\vec E$.

To calculate the potential energy of a dipole, we use
\begin{equation}
U = -\vec{p}\cdot \vec{E}.
\end{equation}
where $\vec p$ is the electric dipole moment and $\vec E$ is the electric field.

We can approximate the field of an electric dipole at $r>>d$ with binomial expansion using
\begin{equation}
    \vec{E}(\vec{r}) \approx \frac{3(\vec{p}\cdot \hat{r})\hat{r} - \vec{p}}{4 \pi \epsilon_0 r^3},
\end{equation}
where $\vec{p}$ is the electric dipole moment, $\hat r$ is the unit vector in the direction or $\vec r$, and $\epsilon_0 = \SI{8.854e-12}{\frac{C^2}{N \cdot m^2}}$ (we probably don't need to know this formula for the exam tbh).


%\begin{figure}[h]
%    \centering
%    \includegraphics[width = 0.5\linewidth]{chapter1_conductors.png}
%    \caption{Image of charging by conduction}
%    \label{figure:1} 
%\end{figure}
