\documentclass[11pt, letterpaper, titlepage]{report}
\usepackage[utf8]{inputenc}
\usepackage{geometry}
\usepackage{color,graphicx,overpic} 
\usepackage{fancyhdr} % header/footer stuff
\usepackage{amssymb} % math symbols
\usepackage{mathtools} % more math stuff
\usepackage{siunitx} % for SI units, ex. $3.5 ~ \si{kg.s^{-2}}$
\usepackage{hyperref} % for hyperlinks
\usepackage{subfiles}

% Page geometry
\geometry{top=2.54cm, left=2.1cm, right=2.1cm, bottom=2.1cm}

% Indentation/space between paragraphs
\setlength{\headheight}{12pt}
\renewcommand{\baselinestretch}{1.5} 
\setlength{\parskip}{0.3cm}
\setlength{\parindent}{0.6cm}

% Line spacing
\renewcommand{\baselinestretch}{1.5	} 

% Title page
\title{\textbf{\Huge{ \begin{center}
Physics 230 Notes\\ % Document name
\end{center} }}}
\author{ Lora Ma and Benjamin Kong\\}

% Header/Footer
\pagestyle{fancy}
\fancyhf{}
\rhead{\thepage}
\lhead{Header}
\rfoot{}

% Hyperlink colors
\hypersetup{
    colorlinks=true,
    linkcolor=blue,
    filecolor=blue,      
    urlcolor=blue,
}

% $$$$$$$$$$$$$ CUSTOM COMMANDS $$$$$$$$$$$$$ %
	% Format for creating new command: \newcommand{name}[num][default]{definition} 
\DeclareMathOperator{\di}{d\!} % derivative operator symbol, ex. $\int f(x) \di x$
\newcommand{\Eval}[3]{\left.#1\right\rvert_{#2}^{#3}} % evaluation bar (ex. evaluating integral or $\Eval{F(x)}{0}{2}$)
\newcommand{\barrows}{\textcolor{magenta}{\Longrightarrow}\quad} % long arrow (start of line)
\newcommand{\barrow}{\quad\textcolor{magenta}{\Longrightarrow}\quad} %long arrow 
\newcommand{\sumi}[1][1]{ \sum_{n={#1}}^{\infty} } % sum starting at n = 1
\newcommand{\limi}[1][n]{ \lim_{{#1}\to\infty} } % limit as n goes to infinity
\newcommand{\ddx}[1][x]{ \dfrac{\text{d}}{\di #1} } % derivative operator
\newcommand{\dyd}[2][y]{ \dfrac{\di #1}{\di #2} } % derivative of y w.r.t. x, (ex. \dyd{x} or \dyd[g]{t}).
\newcommand{\bracks}[1]{ \left( #1 \right) } % scaled left and right brackets
\newcommand{\pfrac}[2]{\left(\frac{#1}{#2}\right)} % scaled fraction with left/right brackets
\newcommand{\tpfrac}[2]{\left(\tfrac{#1}{#2}\right)} % tiny fraction with left/right brackets
\newcommand{\abs}[1]{\left| #1 \right|} % scaled fraction with left/right brackets
\newcommand{\norm}[1]{\left\| #1 \right\|} % scaled fraction with left/right brackets
\newcommand{\abracks}[1]{\langle #1 \rangle} % angled l/r brackets for vectors
\newcommand{\cbracks}[1]{\left\{ #1 \right\}} % scaled curly l/r brackets 
\newcommand{\sbracks}[1]{\left[ #1 \right]} % scaled square l/r brackets 
% $$$$$$$$$$$$$$$$$$$$$$$$$$$$$$$$$$$$$$$$$$$ %

% $$$$$$$$$$$$$ CUSTOM SYMBOLS $$$$$$$$$$$$$$ %
\newcommand{\R}{\mathbb{R}} % Real numbers
\newcommand{\infnums}{1,2,\ldots}
% $$$$$$$$$$$$$$$$$$$$$$$$$$$$$$$$$$$$$$$$$$$ %

\makeatletter
\fancypagestyle{mypagestyle}{%
  \fancyhf{}% Clear header/footer
  \fancyfoot[C]{\thepage}% Page # in middle/centre of footer
  \renewcommand{\headrulewidth}{0pt}% .4pt header rule
  \def\headrule{{\if@fancyplain\let\headrulewidth\plainheadrulewidth\fi
    \color{red}\hrule\@height\headrulewidth\@width\headwidth \vskip-\headrulewidth}}
  \renewcommand{\footrulewidth}{0pt}%
}
\makeatother
\pagestyle{mypagestyle}

\begin{document}

\maketitle

\pagenumbering{gobble}
\maketitle % Create the title page first.
{
  \hypersetup{}
  \parskip 0pt
  \tableofcontents
} % Table  of contents next. 
\newpage
\pagenumbering{arabic}
\pagestyle{mypagestyle}


\subfile{Chapters/chap1.tex}
\section{Gauss's Law}
Gauss's law states that the outward flux of the electric field through a closed surface is equal to the total charge inside the surface divided by the electric constant, $\epsilon_0$:
\begin{equation}
\Phi_E \equiv \oint \vec{E}\cdot \di \vec{A} = \frac{Q_{encl}}{\epsilon_0}.
\end{equation}
Let us examine what this equation means.

We define the \textbf{mass flux} through a surface via
\begin{equation}
\Phi_M = \dfrac{\di M}{\di t},
\end{equation}
where $M$ is the mass and $t$ is time. The surface that we pick to evaluate the flow through is sometimes called a \textit{Gaussian surface}. We also always consider a \textit{stationary} situation, where things do not change with time at a given location. Flux is essentially anything measurable that can be transported (ex. mass, momentum, energy, heat, etc.). The \textbf{mass flux} through a surface can be calculated with a surface integral via
\begin{equation}
\Phi_M = \int \rho_m \vec{v} \cdot \di \vec{A},
\end{equation}
where $\rho_m$ is the mass density and $\vec{V}$ is the velocity. For a tilted surface, the mass flux can be calculated as
\begin{equation}
\Phi_M = \rho_M \vec{v} \cdot \vec{A}.
\end{equation}
When no mass is created or destroyed, we have that
\begin{equation}
\Phi_M = \frac{-\di M_{encl}}{\di t}.
\end{equation}
Note that for any \textit{closed surface}, the mass flux is always zero if the flow is stationary and if there is no creation of air, i.e.,
\begin{equation}
\Phi_M = 0.
\end{equation} 
This is analogous to Gauss's law for a closed surface with no charges inside, i.e.,
\begin{equation}
\Phi_E = \oint \vec{E} \cdot \di \vec{A} = \dfrac{Q_\text{encl}}{\epsilon_0} = 0.
\end{equation}

Mass creation is when:
\begin{equation}
\Phi_M + \frac{\di M}{\di t} > 0,
\end{equation}
and mass destruction is when
\begin{equation}
\Phi_M + \frac{\di M}{\di t} < 0.
\end{equation}

Conservation of charge states that
\begin{equation}
\Phi_Q + \frac{\di Q}{\di t} = 0.
\end{equation}

Second Law of thermodynamics states that
\begin{equation}
\Phi_S + \frac{\di S}{\di t} \geq 0.
\end{equation}

Gauss's law and spherical symmetry. Using coulomb's law for one charge
\begin{equation}
\vec{E} = \frac{q \vec{r}}{4 \pi \epsilon_0 r^2}
\end{equation}

For more than one charge, you need to do superposition

Every region totally inside a conductor is neutral. Excess charge all lies on the surface (static). An interior surface of a conductor, surrounding a cavity, only has charge if there is non-zero charge inside the cavity; the two charges cancel

Spherical symmetry 


\subfile{Chapters/chap2.tex}

\chapter{Electric Potential}
\chapter{Capacitance and Dielectrics}
\chapter{Current, Resistance, and Electromotive Force}
\chapter{Direct-Current Circuits}
Now we shall look in more detail at electrical circuits, which have a network of conductors and other devices connected together. Generally, the conductors can be idealized as thin wires, each of which carries a constant current in a stationary situation, and junctions where all the incoming currents get divided up into all the outgoing currents, with no charge of either sign building up at the junction. This was given by \textbf{Kirchhoff's junction rule} or \textbf{current law} or \textbf{first law: Sum of currents into a junction is zero}.

Since in a stationary situation the electric force is a \textbf{conservative force}, with a \textbf{potential} (potential energy per charge, measured in the SI unit of $\SI{1}{V} = \SI{1}{J/C}$) that is defined up to one overall additive constant (so that the potential differences or voltages between two points are uniquely defined), we get \textbf{Kirchhoff's loop rule} or \textbf{voltage law} or \textbf{second law: the algebraic sum of potential differences around any closed loop is zero} (the algebraic sum means keeping track of signs. If we have a potential increase from $a$ to $b$, so $V_b - V_a = V^{ba} = -V^{ab} > 0$, the potential difference is positive; if $V_b < V_a$, $V_b - V_a = V^{ba} = -V^{ab} < 0$, this is negative).

Now let us use Kirchhoff's rules to get the effective resistance $R_\text{eff}$ of $N$ resistors, of resistances $R_i$ with $i$ running from 1 to $N$, that is, $R_1, R_2, \ldots, R_N$, when the resistors are connected in \textbf{series} and in \textbf{parallel}. In series, we have 
\begin{equation}
R_\text{eff} = R_\text{eq} = \sum R_i = R_1 + R_2 + \cdots + R_N.
\end{equation}
In parallel, we have
\begin{equation}
\dfrac{1}{R_\text{eff}} = \dfrac{1}{R_\text{eq}} = \sum \dfrac{1}{R_i} = \dfrac{1}{R_1} + \dfrac{1}{R_2} + \cdots + \dfrac{1}{R_N}.
\end{equation}

A wire of cross section $A$ can be considered as a parallel arrangement of wires of cross sections adding up to $A$, which fits $\frac{1}{R} = \frac{A}{\rho L} = \frac{\sigma}{L}A$, adding up the $A$'s.

\chapter{Magnetic Field and Magnetic Forces}
\chapter{Sources of Magnetic Field}
\chapter{Electromagnetic Induction}
\chapter{Inductance}
\chapter{Electromagnetic Waves}


\end{document}

