\documentclass[11pt, letterpaper, titlepage]{report}
\usepackage[utf8]{inputenc}
\usepackage{geometry}
\usepackage{color,graphicx,overpic,wrapfig} 
\usepackage{fancyhdr} % header/footer stuff
\usepackage{amssymb} % math symbols
\usepackage{mathtools} % more math stuff
\usepackage{siunitx} % for SI units, ex. $3.5 ~ \si{kg.s^{-2}}$
\usepackage{hyperref} % for hyperlinks
\usepackage{subfiles}

% Page geometry
\geometry{top=2.54cm, left=2.1cm, right=2.1cm, bottom=2.1cm}

% Indentation/space between paragraphs
\setlength{\headheight}{12pt}
\renewcommand{\baselinestretch}{1.5} 
\setlength{\parskip}{0.3cm}
\setlength{\parindent}{0.6cm}

% Title page
\title{\textbf{\Huge{ \begin{center}
Physics 230 Notes\\ % Document name
\end{center} }}}
\author{ Lora Ma and Benjamin Kong\\}

% Header/Footer
\pagestyle{fancy}
\rhead{\thepage}
\lhead{Header}
\rfoot{}

% Hyperlink colors
\hypersetup{
    colorlinks=true,
    linkcolor=blue,
    filecolor=blue,      
    urlcolor=blue,
}

	% Format for creating new command: \newcommand{name}[num][default]{definition} 
\DeclareMathOperator{\di}{d\!} % derivative operator symbol, ex. $\int f(x) \di x$
\newcommand{\Eval}[3]{\left.#1\right\rvert_{#2}^{#3}} % evaluation bar (ex. evaluating integral or $\Eval{F(x)}{0}{2}$)
\newcommand{\barrows}{\textcolor{magenta}{\Longrightarrow}\quad} % long arrow (start of line)
\newcommand{\barrow}{\quad\textcolor{magenta}{\Longrightarrow}\quad} %long arrow 
\newcommand{\sumi}[1][1]{ \sum_{n={#1}}^{\infty} } % sum starting at n = 1
\newcommand{\limi}[1][n]{ \lim_{{#1}\to\infty} } % limit as n goes to infinity
\newcommand{\ddx}[1][x]{ \dfrac{\text{d}}{\di #1} } % derivative operator
\newcommand{\dyd}[2][y]{ \dfrac{\di #1}{\di #2} } % derivative of y w.r.t. x, (ex. \dyd{x} or \dyd[g]{t}).
\newcommand{\bracks}[1]{ \left( #1 \right) } % scaled left and right brackets
\newcommand{\pfrac}[2]{\left(\frac{#1}{#2}\right)} % scaled fraction with left/right brackets
\newcommand{\tpfrac}[2]{\left(\tfrac{#1}{#2}\right)} % tiny fraction with left/right brackets
\newcommand{\abs}[1]{\left| #1 \right|} % scaled fraction with left/right brackets
\newcommand{\norm}[1]{\left\| #1 \right\|} % scaled fraction with left/right brackets
\newcommand{\abracks}[1]{\langle #1 \rangle} % angled l/r brackets for vectors
\newcommand{\cbracks}[1]{\left\{ #1 \right\}} % scaled curly l/r brackets 
\newcommand{\sbracks}[1]{\left[ #1 \right]} % scaled square l/r brackets 
% $$$$$$$$$$$$$$$$$$$$$$$$$$$$$$$$$$$$$$$$$$$ %

% $$$$$$$$$$$$$ CUSTOM SYMBOLS $$$$$$$$$$$$$$ %
\newcommand{\R}{\mathbb{R}} % Real numbers
\newcommand{\infnums}{1,2,\ldots}

\newcommand{\Bv}{ \vec{B} } 
\newcommand{\Ev}{ \vec{E} } 
\newcommand{\Fv}{ \vec{F} } 
\newcommand{\Jv}{ \vec{J} } 
\newcommand{\vv}{ \vec{v} } 
% $$$$$$$$$$$$$$$$$$$$$$$$$$$$$$$$$$$$$$$$$$$ %

\makeatletter
\fancypagestyle{mypagestyle}{%
  \fancyhf{}% Clear header/footer
  \fancyfoot[C]{\thepage}% Page # in middle/centre of footer
  \renewcommand{\headrulewidth}{0pt}% .4pt header rule
  \def\headrule{{\if@fancyplain\let\headrulewidth\plainheadrulewidth\fi
    \color{red}\hrule\@height\headrulewidth\@width\headwidth \vskip-\headrulewidth}}
  \renewcommand{\footrulewidth}{0pt}%
}
\makeatother
\pagestyle{mypagestyle}

\begin{document}

\maketitle

\pagenumbering{gobble}
\maketitle % Create the title page first.
{
  \hypersetup{}
  \parskip 0pt
  \tableofcontents
} % Table  of contents next. 
\newpage
\pagenumbering{arabic}
\pagestyle{mypagestyle}

\subfile{Chapters/chap6.tex}

\subfile{Chapters/chap7.tex}

\subfile{Chapters/chap8.tex}

\chapter{Electromagnetic Induction}

\section{Summary of Electromagnetic Induction}
\begin{itemize}
	\item Faraday's law:
	\begin{equation}
	\varepsilon = -\frac{\di \Phi_B}{\di t} 
	\end{equation}
	for the induced emf in a closed loop from the time rate of charge of magnetic flux through the loop.
	
	\item Lenz's law: The sign in Faraday's law is such that the induced current tends to oppose the change in flux that produced it (stability).
	
	\item If the conductor loop moves in a static magnetic field,
	\begin{equation}
	\varepsilon = \oint (\vec{v} \times \Bv) \cdot \di \vec{l} = -\dfrac{\di \Phi_B}{\di t} = -\dfrac{\di}{\di t} \oint \Bv\cdot\di\vec{A} = -\oint\Bv\cdot\dfrac{\di \vec{A}}{\di t} 
	\end{equation}
	is a motional emf.
	
	\item If $\frac{\di \Bv}{\di t} \neq 0$, $\oint \Ev \cdot \di \vec{l} = -\oint\frac{\di \Bv}{\di t} \cdot \di \vec{A} \neq 0$ makes the electric field nonconservative, so $\Ev \neq -\vec{\nabla} V$.
	
	\item Ampere's law as corrected by Maxwell includes a displacement current $i_D = \varepsilon_0 \frac{\di \Phi_E}{\di t}$: $\oint \Bv \cdot \vec{l} = \mu_0 \int \bracks{\vec{J} + \varepsilon_0 \frac{\di \Ev}{\di t}} \cdot \di \vec{A} = \mu_0(i_c + i_D)_\text{encl}$.
	
	\item The other 3 Maxwell equations are
	\begin{align}
	&\oint \Ev \cdot \di \vec{A} = \dfrac{Q_\text{encl}}{\varepsilon_0}, \\
	&\oint \Bv \cdot \di \vec{A} = 0, \\
	&\oint \Ev \cdot \di \vec{l} = -\dfrac{\di \Phi_B}{\di t}.
	\end{align}
	
	\item When $Q=0$, $J=0$, symmetric under $\Ev \to c\Bv$, $\Bv \to -\frac{1}{c}\Ev$.
	
	\item In 4D relativistic notation, the 4 Maxwell equations can be reduced to $\di F = 0$ and $\delta F = J$, or, using $F = \di A$ to solve $\di F = 0$, $\delta \di A = J$.

\end{itemize}

\chapter{Inductance}
\chapter{Electromagnetic Waves}


\end{document}

